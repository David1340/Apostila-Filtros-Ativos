
\chapter{Problema da aproximação}
\label{chap: aproximacao}

O problema da aproximação é um conceito fundamental na teoria de controle e na análise de sistemas dinâmicos. Ele se refere à tarefa de encontrar uma representação matemática que se aproxime de maneira satisfatória o comportamento de um sistema real. Essa representação pode ser uma função, um modelo matemático ou uma estrutura de controle.

\section{Butterworth}

\section{Chebyshev}

\section{Elíptico}

A aproximação Elíptico também conhecida como aproximação de Cauer caracteriza-se por possui oscilações tanto na faixa de passagem quanto na faixa de rejeição. Em troca, ele obtém uma queda mais acentuada na faixa de transição, o que faz com que a aproximação de Cauer seja capaz de satisfazer as especificações de projeto de filtros com uma ordem menor do que seria obtidos se fossem usadas as aproximações de Butterworth e de Chebyshev. Na tabela \ref{tab:eliptico} são apresentados exemplos de funções de transferência de aproximações de filtros passa-baixas do tipo elíptico com $A_\text{max} = 0.5$ dB e ordem 2. 

\begin{table}[!h]
    \centering
    \renewcommand{\arraystretch}{2.0}  % aumenta o espaçamento entre as linhas
    \caption{Exemplos de funções de transferência de filtros elípticos para $A_\text{max}  = 0.5$ dB de ordem 2.}
    \begin{tabular}{|c|c|c|}
        \hline
        $\Omega_s$ & $1/H(s)$ & $A_\text{min}$ (dB) \\
        \hline 
        1.5 & $ \frac{0.38540(s^2 + 3.92705)}{s^2 + 1.03153s + 1.60319}$ & 8.3\\
        \hline
        2.0 & $\frac{0.20133(s^2 + 7.4641)}{s^2 +  1.24504s +  1.59179}$& 13.9 \\
        \hline
        3.0 & $\frac{0.083974(s^2 + 17.48528)}{s^2 +  1.35715s +  1.55532}$& 21.5 \\ 
        \hline       
    \end{tabular} 
    \label{tab:eliptico}
\end{table}

\section{Bessel}

As aproximações anteriores são focadas no ganho ou atenuação do filtro, porém em algumas aplicações como sistemas de transmissão digitais, a distorção de fase é um fator importante a ser considerado. A aproximação de Bessel é projetada para obter uma curva de fase tão plana quanto possível na faixa de passagem.

A aproximação de Bessel normalziada é dada por:

\begin{equation}
    H(s) = \frac{B_n(0)}{B_n(s)}
\end{equation}
em que $B_n(s)$ é enésimo termo do polinômio de Bessel o qual é definido pela forma recursiva:

\begin{equation}
    B_0(s) = 1 
\end{equation}

\begin{equation}
    B_1(s) = s + 1
\end{equation}

\begin{equation}
    B_n(s) = (2n -1)B_{n-1}(s) + s^2B_{n-2}(s)
\end{equation}